 
%%
%% This is file `elsarticle-template-num.tex',
%% generated with the docstrip utility.
%%
%% The original source files were:
%%
%% elsarticle.dtx  (with options: `numtemplate')
%% 
%% Copyright 2007, 2008 Elsevier Ltd.
%% 
%% This file is part of the 'Elsarticle Bundle'.
%% -------------------------------------------
%% 
%% It may be distributed under the conditions of the LaTeX Project Public
%% License, either version 1.2 of this license or (at your option) any
%% later version.  The latest version of this license is in
%%    http://www.latex-project.org/lppl.txt
%% and version 1.2 or later is part of all distributions of LaTeX
%% version 1999/12/01 or later.
%% 
%% The list of all files belonging to the 'Elsarticle Bundle' is
%% given in the file `manifest.txt'.
%% 

%% Template article for Elsevier's document class `elsarticle'
%% with numbered style bibliographic references
%% SP 2008/03/01

%\documentclass[preprint,12pt]{elsarticle}

%% Use the option review to obtain double line spacing
\documentclass[preprint,review,12pt]{elsarticle}

%% Use the options 1p,twocolumn; 3p; 3p,twocolumn; 5p; or 5p,twocolumn
%% for a journal layout:
%% \documentclass[final,1p,times]{elsarticle}
%% \documentclass[final,1p,times,twocolumn]{elsarticle}
%% \documentclass[final,3p,times]{elsarticle}
%% \documentclass[final,3p,times,twocolumn]{elsarticle}
%% \documentclass[final,5p,times]{elsarticle}
%% \documentclass[final,5p,times,twocolumn]{elsarticle}

%% if you use PostScript figures in your article
%% use the graphics package for simple commands
%% \usepackage{graphics}
%% or use the graphicx package for more complicated commands
%% \usepackage{graphicx}
%% or use the epsfig package if you prefer to use the old commands
%% \usepackage{epsfig}

\pdfoutput=1

%% The amssymb package provides various useful mathematical symbols
\usepackage{amssymb}
%% The amsthm package provides extended theorem environments
%% \usepackage{amsthm}

%\usepackage{hyperref} %generate bookmarks for better navigation

%%%%%%%%%%%%%%%%%%%%%%%%%%%%%%%%%%%%%%%%%%%%%%%%%%%%%%%%%%%%%%%%%%%%%%%%%%%%%
%%% author packages
%%%%%%%%%%%%%%%%%%%%%%%%%%%%%%%%%%%%%%%%%%%%%%%%%%%%%%%%%%%%%%%%%%%%%%%%%%%%%
%\usepackage{times}
\usepackage{epsfig}
\usepackage[ruled]{algorithm2e}
\usepackage{algorithmic}
\usepackage{subfigure}
\usepackage{xspace}
\usepackage{rotating}
\usepackage{fancyhdr}
\usepackage{url}
\usepackage{color}
\usepackage{booktabs}
\usepackage{amsmath}  
\usepackage{amssymb}
\DeclareFontFamily{OT1}{pzc}{}
\DeclareFontShape{OT1}{pzc}{m}{it}%
              {<-> s * [0.900] pzcmi7t}{}
\DeclareMathAlphabet{\mathpzc}{OT1}{pzc}%
                                 {m}{it}
                                 
\urldef{\mailsa}\path|{lwei, eribeiro}@fit.edu|
%\usepackage{bm}
%\usepackage{eucal}
%
%\DeclareMathAlphabet{\mathpzc}{OT1}{pzc}{m}{it}

%%%%%%%%%%%%%%%%%%%%%%%%%%%%%%%%%%%%%%%%%%%%%%%%%%%%%%%%%%%%%%%%%%%%%%%%%%%%%
%%% Authors' definitions
%%%%%%%%%%%%%%%%%%%%%%%%%%%%%%%%%%%%%%%%%%%%%%%%%%%%%%%%%%%%%%%%%%%%%%%%%%%%%
\newcommand{\argmax}{\operatornamewithlimits{arg\,\,max}}
\newcommand{\argmin}{\operatornamewithlimits{arg\,\,min}}

\renewcommand{\baselinestretch}{2.0}

\newcommand{\etal}{{et al.}}
\newcommand{\ie}{{i.e., }}          
\newcommand{\eg}{{e.g., }}          


\def\xprime{x^\prime}
\def\yprime{y^\prime}
\def\Transpose{{^\mathsf{T}}}


% using package soul instead of ulem
\usepackage{soul}
\newcommand{\remove}[1]{{\bf \textcolor{red}{\st{#1}}}}
\newcommand{\add}[1]{{\bf \textcolor{blue}{{\sf #1}}}}
\newcommand{\ask}[1]{{\textcolor{cyan}{{\sf #1}}}\marginpar{$\leftarrow$ \sf question}}
\newcommand{\suggestion}[1]{{\sf #1}\marginpar{$\leftarrow$ \sf suggestion}}
\newcommand{\comment}[1]{{\bf \textcolor{red}{{\sf #1}}}}
% margin note
\newcommand{\mn}[1]{\marginpar{{\small \sf #1}}}
%%%
\newcommand{\Visual}{\mathcal{V}}                     % bold \mathcal{I} letter

\newcommand{\thefrench}{G\^{a}teaux }
\newcommand{\MovingGroup}{M}
\newcommand{\FixedGroup}{F}
\newcommand{\NBin}{B}										%number of bins in the histogram
\newcommand{\Prob}{P}										%probability function
\newcommand{\Entropy}{H}
\newcommand{\JSD}{\text{JS}}						%Jesen-Shannon divergence
\newcommand{\bfx}{\mathbf{x}}						%3-D coordinate vector
\newcommand{\SimFunc}{\text{Sim}}
\newcommand{\Ecorr}{E^{\text{corr}}}		%Correspondence energy

\newtheorem{theorem}{Theorem}[section]
\newtheorem{lemma}[theorem]{Lemma}
\newenvironment{proof}[1][Proof]{\begin{trivlist}
\item[\hskip \labelsep {\bfseries #1}]}{\end{trivlist}}
\newenvironment{definition}[1][Definition]{\begin{trivlist}
\item[\hskip \labelsep {\bfseries #1}]}{\end{trivlist}}
\newenvironment{example}[1][Example]{\begin{trivlist}
\item[\hskip \labelsep {\bfseries #1}]}{\end{trivlist}}
\newenvironment{remark}[1][Remark]{\begin{trivlist}
\item[\hskip \labelsep {\bfseries #1}]}{\end{trivlist}}

\journal{Computer Vision and Image Understanding}

\begin{document}


\include{notationsymbols}

\begin{frontmatter}



\title{Intergroup Registration with Jensen-Shannon Demons}

% {Paper Number 513}
\author{Wei Liu}
\address{
}

\begin{abstract}
doodle
\end{abstract}

\begin{keyword}


\end{keyword}

\end{frontmatter}

%% \linenumbers

%% main text
%------------------------------------------------------------------------- 
\section{Introduction}



\section{Jesen-Shannon Demons}

Given a group of images $I=\left\{I_s(\bfx) |\,\, s=1\ldots N, \bfx\in \mathbb{R}^3\right\}$, we use a group-wise registration algorithm~\cite{Wu20111968} to align the images. Then, we estimate a histogram field $\FixedGroup(\bfx)$ from the aligned images, where each ``voxel'' of $\FixedGroup(\bfx)$ is a probability density function (PDF). In this paper, we simply represent the PDFs using histograms, \ie each ``voxel'' $\FixedGroup(\bfx) = \left\{ p_1(\bfx), p_2(\bfx), \cdots, p_\NBin(\bfx) \right\}^\Transpose$ is a vector, where $\NBin \in \mathbb{N}$ is the number of bins in the histogram and $\sum_{i=1}^\NBin p_i(\bfx)=1$ for all $\bfx$. After estimating a histogram field for each group, we formulate the inter-group registration propblem as to align the corresponding histogram fields, inconstrast to existing approaches (citations) that simply register the group mean images.

We use $\FixedGroup$ and $\MovingGroup$ to represent the ``fixed'' and ``moving'' histogram fields, and the goal is to find the spatial transformation $s$ such that a similarity criterion $\SimFunc(\FixedGroup,\MovingGroup\circ s)$ is minimized~\cite{Vercauteren2009S61}. There are many choices for $\SimFunc$. For instance, mean-squared error (MSE) is used as similarity criterion for intensity-based registration. In~\cite{peyrat_multichannel_demons}, MSE is used for multi-channel Demons registration. In principle, we can treat histogram fields as ordinary multi-channel images, and use the approach in~\cite{peyrat_multichannel_demons} to align them. However, MSE can be sensitive to outliers. Instead, we resort to Jesen-Shannon divergence~\cite{wang_vemuri_js_divergence}, which is a distance measure tailored to probability density functions and is robust to outliers.

Given two probability distributions $\Prob_1,\Prob_2$, the Jesen-Shannon divegence is defined as follows~\cite{wang_vemuri_js_divergence}:
\begin{align}
	\JSD(\Prob_1,\Prob_2) &= \Entropy\left(\frac{1}{2}\Prob_1+\frac{1}{2}\Prob_2\right)-\frac{1}{2}\Entropy(\Prob_1)-\frac{1}{2}\Entropy(\Prob_2),
	\label{eq:js_divergence}
\end{align}
where $\Entropy$ is the well-known Shannon entropy. For discrete random variable $\mathbf{X}$ with $n$ events ${x_1,\ldots, x_n}$, Shannon entropy is defined as:
\begin{align}
\Entropy\mathbf{X})=-\sum_{i=1}^n p(x_i) \log_b p(x_i).
\label{eq:shannon}
\end{align}
Given the definition of Jesen-Shannon divergence in (\ref{eq:js_divergence}), we define a similarity criterion following the formality of MSE-based criterion~\cite{Vercauteren2009S61}, given by:
\begin{align}
\SimFunc(\FixedGroup,\MovingGroup\circ s)=\frac{1}{|\Omega|} \sum_{p \in \Omega} \JSD\left(\FixedGroup(p),\MovingGroup(s(p))\right).
\label{eq:js_similarity}
\end{align}
Here, $\Omega$ is the region of overlap between $\FixedGroup$ and $\MovingGroup\circ s$.
Again, in a similar fasion to previous Demons methods~\cite{Vercauteren2009S61,peyrat_multichannel_demons}, we can formulate the linearization of the multi-channel Demons correspondence energy as (~\ref{apx:linearization}):
\begin{align}
	\Ecorr = \frac{1}{|\Omega|} \sum_{p \in \Omega}
	\label{eq:linearization}
\end{align}

\appendix

\section{Implementation notes}

\subsection{Shannon Entropy of Histograms}

Probability $p(x_i)$ is estimated using a histogram as:
\begin{align}
p(x_i)\approx\frac{h_i}{C}, \hspace{.5in} \text{with} \hspace{.5in} C=\sum_{i=1}^{n} h_i
\label{eq:histogram}
\end{align}
with $h_i$ as count of events $x_i$ and $C$ is the total number of trials. Substituting (\ref{eq:histogram}) into (\ref{eq:shannon}), we have:
\begin{align}
	H(\mathbf{X})&\approx - \sum_{i=1}^n \frac{h_i}{C} \log_b \frac{h_i}{C} \notag\\
	&=-\frac{1}{C} \sum_{i=1}^n \left( h_i \log_b h_i - h_i \log_b C \right) \notag \\
	&=-\frac{1}{C} \sum_{i=1}^n h_i \log_b h_i + \frac{1}{C}\sum_{i=1}^n h_i \log_b C \notag \\
	&= -\frac{1}{C} \sum_{i=1}^n h_i \log_b h_i + \log_b C .
\end{align}


\section{Linearization of J-S divergence}
\label{apx:linearization}
In this section, we explain the details of converting our similarity criterion based on Jensen-Shannon divergence (Equation~\ref{eq:js_similarity}) into the linearized correspondence energy (Equation~\ref{eq:linearization}).
Here, we largely follow the derivation and formality in~\cite{Vercauteren2009S61}.

{\small 
\bibliographystyle{unsrt}
\bibliography{wei_intergroup}
}



\end{document}

\endinput
%%
%% End of file `elsarticle-template-num.tex'.









